%% Generated by Sphinx.
\def\sphinxdocclass{report}
\documentclass[letterpaper,10pt,french]{sphinxmanual}
\ifdefined\pdfpxdimen
   \let\sphinxpxdimen\pdfpxdimen\else\newdimen\sphinxpxdimen
\fi \sphinxpxdimen=.75bp\relax
\ifdefined\pdfimageresolution
    \pdfimageresolution= \numexpr \dimexpr1in\relax/\sphinxpxdimen\relax
\fi
%% let collapsable pdf bookmarks panel have high depth per default
\PassOptionsToPackage{bookmarksdepth=5}{hyperref}

\PassOptionsToPackage{warn}{textcomp}
\usepackage[utf8]{inputenc}
\ifdefined\DeclareUnicodeCharacter
% support both utf8 and utf8x syntaxes
  \ifdefined\DeclareUnicodeCharacterAsOptional
    \def\sphinxDUC#1{\DeclareUnicodeCharacter{"#1}}
  \else
    \let\sphinxDUC\DeclareUnicodeCharacter
  \fi
  \sphinxDUC{00A0}{\nobreakspace}
  \sphinxDUC{2500}{\sphinxunichar{2500}}
  \sphinxDUC{2502}{\sphinxunichar{2502}}
  \sphinxDUC{2514}{\sphinxunichar{2514}}
  \sphinxDUC{251C}{\sphinxunichar{251C}}
  \sphinxDUC{2572}{\textbackslash}
\fi
\usepackage{cmap}
\usepackage[T1]{fontenc}
\usepackage{amsmath,amssymb,amstext}
\usepackage{babel}



\usepackage{tgtermes}
\usepackage{tgheros}
\renewcommand{\ttdefault}{txtt}



\usepackage[Sonny]{fncychap}
\ChNameVar{\Large\normalfont\sffamily}
\ChTitleVar{\Large\normalfont\sffamily}
\usepackage{sphinx}

\fvset{fontsize=auto}
\usepackage{geometry}


% Include hyperref last.
\usepackage{hyperref}
% Fix anchor placement for figures with captions.
\usepackage{hypcap}% it must be loaded after hyperref.
% Set up styles of URL: it should be placed after hyperref.
\urlstyle{same}

\addto\captionsfrench{\renewcommand{\contentsname}{Contents:}}

\usepackage{sphinxmessages}
\setcounter{tocdepth}{3}
\setcounter{secnumdepth}{3}


\title{Ecosystème logistique}
\date{juin 26, 2021}
\release{}
\author{Judith Bellon, Gabrielle Vernet, César Almecija, Louis\sphinxhyphen{}Justin Tallot}
\newcommand{\sphinxlogo}{\vbox{}}
\renewcommand{\releasename}{}
\makeindex
\begin{document}

\ifdefined\shorthandoff
  \ifnum\catcode`\=\string=\active\shorthandoff{=}\fi
  \ifnum\catcode`\"=\active\shorthandoff{"}\fi
\fi

\pagestyle{empty}
\sphinxmaketitle
\pagestyle{plain}
\sphinxtableofcontents
\pagestyle{normal}
\phantomsection\label{\detokenize{index::doc}}



\chapter{Interface Homme\sphinxhyphen{}Machine}
\label{\detokenize{index:interface-homme-machine}}

\section{Interface complète}
\label{\detokenize{index:module-src.ihm.ihm_complet}}\label{\detokenize{index:interface-complete}}\index{module@\spxentry{module}!src.ihm.ihm\_complet@\spxentry{src.ihm.ihm\_complet}}\index{src.ihm.ihm\_complet@\spxentry{src.ihm.ihm\_complet}!module@\spxentry{module}}\index{Wind (classe dans src.ihm.ihm\_complet)@\spxentry{Wind}\spxextra{classe dans src.ihm.ihm\_complet}}

\begin{fulllineitems}
\phantomsection\label{\detokenize{index:src.ihm.ihm_complet.Wind}}\pysigline{\sphinxbfcode{\sphinxupquote{class }}\sphinxcode{\sphinxupquote{src.ihm.ihm\_complet.}}\sphinxbfcode{\sphinxupquote{Wind}}}
\sphinxAtStartPar
Classe contenant l’interface Homme\sphinxhyphen{}Machine pour le projet.
\index{appui\_bouton\_OK() (méthode src.ihm.ihm\_complet.Wind)@\spxentry{appui\_bouton\_OK()}\spxextra{méthode src.ihm.ihm\_complet.Wind}}

\begin{fulllineitems}
\phantomsection\label{\detokenize{index:src.ihm.ihm_complet.Wind.appui_bouton_OK}}\pysiglinewithargsret{\sphinxbfcode{\sphinxupquote{appui\_bouton\_OK}}}{}{{ $\rightarrow$ None}}
\sphinxAtStartPar
Listener pour le bouton ok.
Prépare les données pour lancer la clusterisation et l’affichage de la carte.
cf. lancement\_clustering

\end{fulllineitems}

\index{lancement\_clustering() (méthode src.ihm.ihm\_complet.Wind)@\spxentry{lancement\_clustering()}\spxextra{méthode src.ihm.ihm\_complet.Wind}}

\begin{fulllineitems}
\phantomsection\label{\detokenize{index:src.ihm.ihm_complet.Wind.lancement_clustering}}\pysiglinewithargsret{\sphinxbfcode{\sphinxupquote{lancement\_clustering}}}{}{{ $\rightarrow$ None}}
\sphinxAtStartPar
Lance la clusterisation à l’aide des paramètres entrés par l’utilisateur.
Ensuite, affiche la carte.

\end{fulllineitems}


\end{fulllineitems}



\section{Interaction avec l’utilisateur \sphinxstyleemphasis{via} fichier CSV}
\label{\detokenize{index:module-src.ihm.ihm_csv}}\label{\detokenize{index:interaction-avec-l-utilisateur-via-fichier-csv}}\index{module@\spxentry{module}!src.ihm.ihm\_csv@\spxentry{src.ihm.ihm\_csv}}\index{src.ihm.ihm\_csv@\spxentry{src.ihm.ihm\_csv}!module@\spxentry{module}}
\sphinxAtStartPar
Première interface Homme\sphinxhyphen{}machine : utilisation d’un tableau \sphinxtitleref{CSV} pour récupérer les informations données par l’utilisateur

\sphinxAtStartPar
Paramètres modifiables dans la fonction \sphinxcode{\sphinxupquote{clusterize}} : le nombre de clusters

\sphinxAtStartPar
TODO : Paramètres modifiables souhaités en plus : encadrement du nombre de clusters, taille des clusters


\section{Utilitaire : fenêtre d’accueil pour \sphinxstyleliteralintitle{\sphinxupquote{ihm\_complet}}}
\label{\detokenize{index:module-src.ihm.ihm_pyqt}}\label{\detokenize{index:utilitaire-fenetre-d-accueil-pour-ihm-complet}}\index{module@\spxentry{module}!src.ihm.ihm\_pyqt@\spxentry{src.ihm.ihm\_pyqt}}\index{src.ihm.ihm\_pyqt@\spxentry{src.ihm.ihm\_pyqt}!module@\spxentry{module}}\index{InputFenetre (classe dans src.ihm.ihm\_pyqt)@\spxentry{InputFenetre}\spxextra{classe dans src.ihm.ihm\_pyqt}}

\begin{fulllineitems}
\phantomsection\label{\detokenize{index:src.ihm.ihm_pyqt.InputFenetre}}\pysigline{\sphinxbfcode{\sphinxupquote{class }}\sphinxcode{\sphinxupquote{src.ihm.ihm\_pyqt.}}\sphinxbfcode{\sphinxupquote{InputFenetre}}}
\sphinxAtStartPar
Le widget qui permet à l’utilisateur de rentrer les paramètres de clustering

\end{fulllineitems}



\section{Obsolète ouverture d’un fichier \sphinxstyleliteralintitle{\sphinxupquote{HTML}} dans un navigateur Web :}
\label{\detokenize{index:obsolete-ouverture-d-un-fichier-html-dans-un-navigateur-web}}\phantomsection\label{\detokenize{index:module-src.ihm.web}}\index{module@\spxentry{module}!src.ihm.web@\spxentry{src.ihm.web}}\index{src.ihm.web@\spxentry{src.ihm.web}!module@\spxentry{module}}\index{open\_html() (dans le module src.ihm.web)@\spxentry{open\_html()}\spxextra{dans le module src.ihm.web}}

\begin{fulllineitems}
\phantomsection\label{\detokenize{index:src.ihm.web.open_html}}\pysiglinewithargsret{\sphinxcode{\sphinxupquote{src.ihm.web.}}\sphinxbfcode{\sphinxupquote{open\_html}}}{\emph{\DUrole{n}{adresse}}}{}
\sphinxAtStartPar
Affichage du html depuis python. Il faut être dans le répertoire ihm pour le lancer.
\begin{quote}\begin{description}
\item[{Paramètres}] \leavevmode
\sphinxAtStartPar
\sphinxstyleliteralstrong{\sphinxupquote{adresse}} \textendash{} l’adresse du fichier à ouvrir

\end{description}\end{quote}

\end{fulllineitems}



\chapter{Clusterizer}
\label{\detokenize{index:clusterizer}}

\section{Fichier principal}
\label{\detokenize{index:module-src.clusterizer.clusterizer}}\label{\detokenize{index:fichier-principal}}\index{module@\spxentry{module}!src.clusterizer.clusterizer@\spxentry{src.clusterizer.clusterizer}}\index{src.clusterizer.clusterizer@\spxentry{src.clusterizer.clusterizer}!module@\spxentry{module}}\index{calcule\_nb\_clusters\_par\_zone() (dans le module src.clusterizer.clusterizer)@\spxentry{calcule\_nb\_clusters\_par\_zone()}\spxextra{dans le module src.clusterizer.clusterizer}}

\begin{fulllineitems}
\phantomsection\label{\detokenize{index:src.clusterizer.clusterizer.calcule_nb_clusters_par_zone}}\pysiglinewithargsret{\sphinxcode{\sphinxupquote{src.clusterizer.clusterizer.}}\sphinxbfcode{\sphinxupquote{calcule\_nb\_clusters\_par\_zone}}}{\emph{\DUrole{n}{liste\_df}}, \emph{\DUrole{n}{nb\_clusters}}}{}
\sphinxAtStartPar
TODO
\begin{quote}\begin{description}
\item[{Paramètres}] \leavevmode\begin{itemize}
\item {} 
\sphinxAtStartPar
\sphinxstyleliteralstrong{\sphinxupquote{liste\_df}} \textendash{} 

\item {} 
\sphinxAtStartPar
\sphinxstyleliteralstrong{\sphinxupquote{nb\_clusters}} \textendash{} 

\end{itemize}

\item[{Renvoie}] \leavevmode
\sphinxAtStartPar


\end{description}\end{quote}

\end{fulllineitems}

\index{clusterize() (dans le module src.clusterizer.clusterizer)@\spxentry{clusterize()}\spxextra{dans le module src.clusterizer.clusterizer}}

\begin{fulllineitems}
\phantomsection\label{\detokenize{index:src.clusterizer.clusterizer.clusterize}}\pysiglinewithargsret{\sphinxcode{\sphinxupquote{src.clusterizer.clusterizer.}}\sphinxbfcode{\sphinxupquote{clusterize}}}{\emph{\DUrole{n}{df}\DUrole{p}{:} \DUrole{n}{pandas.core.frame.DataFrame}}, \emph{\DUrole{n}{k}\DUrole{p}{:} \DUrole{n}{int}}, \emph{\DUrole{n}{column\_geometry}\DUrole{p}{:} \DUrole{n}{str} \DUrole{o}{=} \DUrole{default_value}{\textquotesingle{}geometry\textquotesingle{}}}, \emph{\DUrole{n}{is\_dict}\DUrole{p}{:} \DUrole{n}{bool} \DUrole{o}{=} \DUrole{default_value}{False}}, \emph{\DUrole{n}{weight}\DUrole{p}{:} \DUrole{n}{bool} \DUrole{o}{=} \DUrole{default_value}{True}}}{{ $\rightarrow$ Tuple\DUrole{p}{{[}}pandas.core.frame.DataFrame\DUrole{p}{, }pandas.core.frame.DataFrame\DUrole{p}{{]}}}}
\sphinxAtStartPar
Clusterise à l’aide de l’algorithme des k\sphinxhyphen{}moyennes. Attention, fait du en\sphinxhyphen{}place.
\begin{quote}\begin{description}
\item[{Paramètres}] \leavevmode\begin{itemize}
\item {} 
\sphinxAtStartPar
\sphinxstyleliteralstrong{\sphinxupquote{df}} \textendash{} La (Geo)DataFrame contenant les points à clusteriser.

\item {} 
\sphinxAtStartPar
\sphinxstyleliteralstrong{\sphinxupquote{k}} \textendash{} Le nombre de clusters à calculer.

\item {} 
\sphinxAtStartPar
\sphinxstyleliteralstrong{\sphinxupquote{column\_geometry}} \textendash{} A spécifier si la colonne contenant les points n’est pas la colonne par défaut (« geometry »)

\item {} 
\sphinxAtStartPar
\sphinxstyleliteralstrong{\sphinxupquote{is\_dict}} \textendash{} Indiquer True si jamais la colonne contenant les points ne contient pas d’objets
shapely.geometry.Points, mais un dictionnaire (en général, lorsque le fichier provient d’un GeoJSON)

\end{itemize}

\item[{Renvoie}] \leavevmode
\sphinxAtStartPar
Deux GeoDataFrame.
Une première GeoDataFrame entrée contenant une colonne en plus (« cluster ») : celle\sphinxhyphen{}ci permet de savoir pour chaque
point le numéro du cluster qui lui a été affecté.
Une deuxième GeoDataFrame contenant les informations détaillées de chaque cluster : centre de masse (« centroids »),
enveloppe convexe (« hulls ») et nombre d’établissements dans le cluster (« taille »)

\end{description}\end{quote}

\end{fulllineitems}

\index{main\_json() (dans le module src.clusterizer.clusterizer)@\spxentry{main\_json()}\spxextra{dans le module src.clusterizer.clusterizer}}

\begin{fulllineitems}
\phantomsection\label{\detokenize{index:src.clusterizer.clusterizer.main_json}}\pysiglinewithargsret{\sphinxcode{\sphinxupquote{src.clusterizer.clusterizer.}}\sphinxbfcode{\sphinxupquote{main\_json}}}{\emph{\DUrole{n}{rayon}\DUrole{p}{:} \DUrole{n}{int} \DUrole{o}{=} \DUrole{default_value}{8}}, \emph{\DUrole{n}{secteur\_NAF}\DUrole{p}{:} \DUrole{n}{List\DUrole{p}{{[}}str\DUrole{p}{{]}}} \DUrole{o}{=} \DUrole{default_value}{\textquotesingle{}\textquotesingle{}}}, \emph{\DUrole{n}{nb\_clusters}\DUrole{p}{:} \DUrole{n}{int} \DUrole{o}{=} \DUrole{default_value}{50}}, \emph{\DUrole{n}{adresse\_map}\DUrole{p}{:} \DUrole{n}{str} \DUrole{o}{=} \DUrole{default_value}{\textquotesingle{}output/clusterized\_map\_seine.html\textquotesingle{}}}, \emph{\DUrole{n}{reduce}\DUrole{p}{:} \DUrole{n}{bool} \DUrole{o}{=} \DUrole{default_value}{False}}, \emph{\DUrole{n}{threshold}\DUrole{p}{:} \DUrole{n}{int} \DUrole{o}{=} \DUrole{default_value}{1000}}}{{ $\rightarrow$ None}}
\sphinxAtStartPar
Fonction principale à exécuter pour successivement ouvrir la DataFrame contenant les données,
nettoyer la DataFrame, filtrer par secteurs NAF, ne garder que les magasins proche du centre de Paris,
séparer par la Seine, clusteriser et sauvegarder dans une carte.
La répartition entre les secteurs de la Seine est calculée automatiquement.
\begin{quote}\begin{description}
\item[{Paramètres}] \leavevmode\begin{itemize}
\item {} 
\sphinxAtStartPar
\sphinxstyleliteralstrong{\sphinxupquote{rayon}} \textendash{} le rayon (à partir du centre de Paris).

\item {} 
\sphinxAtStartPar
\sphinxstyleliteralstrong{\sphinxupquote{secteur\_NAF}} \textendash{} les secteurs NAF à sélectionner.

\item {} 
\sphinxAtStartPar
\sphinxstyleliteralstrong{\sphinxupquote{nb\_clusters}} \textendash{} le nombre de clusters à calculer.

\item {} 
\sphinxAtStartPar
\sphinxstyleliteralstrong{\sphinxupquote{adresse\_map}} \textendash{} l’adresse de la carte en sortie.

\item {} 
\sphinxAtStartPar
\sphinxstyleliteralstrong{\sphinxupquote{reduce}} \textendash{} mettre \sphinxcode{\sphinxupquote{True}} pour n’utiliser qu’une version allégée des données (plus rapide).

\item {} 
\sphinxAtStartPar
\sphinxstyleliteralstrong{\sphinxupquote{threshold}} \textendash{} nombre de données utilisées si reduce= \sphinxcode{\sphinxupquote{True}}

\end{itemize}

\item[{Renvoie}] \leavevmode
\sphinxAtStartPar
\sphinxcode{\sphinxupquote{None}}

\end{description}\end{quote}

\end{fulllineitems}

\index{map\_rapport\_a\_la\_seine() (dans le module src.clusterizer.clusterizer)@\spxentry{map\_rapport\_a\_la\_seine()}\spxextra{dans le module src.clusterizer.clusterizer}}

\begin{fulllineitems}
\phantomsection\label{\detokenize{index:src.clusterizer.clusterizer.map_rapport_a_la_seine}}\pysiglinewithargsret{\sphinxcode{\sphinxupquote{src.clusterizer.clusterizer.}}\sphinxbfcode{\sphinxupquote{map\_rapport\_a\_la\_seine}}}{\emph{\DUrole{n}{args\_tuple}\DUrole{p}{:} \DUrole{n}{Tuple\DUrole{p}{{[}}int\DUrole{p}{, }pandas.core.frame.DataFrame\DUrole{p}{{]}}}}}{{ $\rightarrow$ pandas.core.frame.DataFrame}}
\sphinxAtStartPar
TODO
\begin{quote}\begin{description}
\item[{Paramètres}] \leavevmode
\sphinxAtStartPar
\sphinxstyleliteralstrong{\sphinxupquote{args\_tuple}} \textendash{} 

\item[{Renvoie}] \leavevmode
\sphinxAtStartPar


\end{description}\end{quote}

\end{fulllineitems}

\index{nettoyer() (dans le module src.clusterizer.clusterizer)@\spxentry{nettoyer()}\spxextra{dans le module src.clusterizer.clusterizer}}

\begin{fulllineitems}
\phantomsection\label{\detokenize{index:src.clusterizer.clusterizer.nettoyer}}\pysiglinewithargsret{\sphinxcode{\sphinxupquote{src.clusterizer.clusterizer.}}\sphinxbfcode{\sphinxupquote{nettoyer}}}{\emph{\DUrole{n}{df}\DUrole{p}{:} \DUrole{n}{pandas.core.frame.DataFrame}}, \emph{\DUrole{n}{reduce}\DUrole{p}{:} \DUrole{n}{bool} \DUrole{o}{=} \DUrole{default_value}{False}}, \emph{\DUrole{n}{threshold}\DUrole{p}{:} \DUrole{n}{int} \DUrole{o}{=} \DUrole{default_value}{1000}}, \emph{\DUrole{n}{column\_geometry}\DUrole{p}{:} \DUrole{n}{str} \DUrole{o}{=} \DUrole{default_value}{\textquotesingle{}geometry\textquotesingle{}}}}{{ $\rightarrow$ pandas.core.frame.DataFrame}}
\sphinxAtStartPar
Nettoie la DataFrame. Enlève les na.
Si spécifié, ne retient que les premières données de la DataFrame.
\begin{quote}\begin{description}
\item[{Paramètres}] \leavevmode\begin{itemize}
\item {} 
\sphinxAtStartPar
\sphinxstyleliteralstrong{\sphinxupquote{df}} \textendash{} La DataFrame.

\item {} 
\sphinxAtStartPar
\sphinxstyleliteralstrong{\sphinxupquote{reduce}} \textendash{} Si True, ne prend que les premières données.

\item {} 
\sphinxAtStartPar
\sphinxstyleliteralstrong{\sphinxupquote{threshold}} \textendash{} Dans le cas où reduce=True, nombre de données à sélectionner.

\item {} 
\sphinxAtStartPar
\sphinxstyleliteralstrong{\sphinxupquote{column\_geometry}} \textendash{} A spécifier si la colonne contenant les points n’est pas la colonne par défaut (« geometry »)

\end{itemize}

\item[{Renvoie}] \leavevmode
\sphinxAtStartPar
Une DataFrame nettoyée.

\end{description}\end{quote}

\end{fulllineitems}

\index{process\_rapport\_a\_la\_seine() (dans le module src.clusterizer.clusterizer)@\spxentry{process\_rapport\_a\_la\_seine()}\spxextra{dans le module src.clusterizer.clusterizer}}

\begin{fulllineitems}
\phantomsection\label{\detokenize{index:src.clusterizer.clusterizer.process_rapport_a_la_seine}}\pysiglinewithargsret{\sphinxcode{\sphinxupquote{src.clusterizer.clusterizer.}}\sphinxbfcode{\sphinxupquote{process\_rapport\_a\_la\_seine}}}{\emph{\DUrole{n}{no\_zone}\DUrole{p}{:} \DUrole{n}{int}}, \emph{\DUrole{n}{df}\DUrole{p}{:} \DUrole{n}{pandas.core.frame.DataFrame}}, \emph{\DUrole{n}{shared\_array}\DUrole{p}{:} \DUrole{n}{multiprocessing.context.BaseContext.Array}}}{{ $\rightarrow$ None}}
\sphinxAtStartPar
TODO
\begin{quote}\begin{description}
\item[{Paramètres}] \leavevmode\begin{itemize}
\item {} 
\sphinxAtStartPar
\sphinxstyleliteralstrong{\sphinxupquote{no\_zone}} \textendash{} 

\item {} 
\sphinxAtStartPar
\sphinxstyleliteralstrong{\sphinxupquote{df}} \textendash{} 

\item {} 
\sphinxAtStartPar
\sphinxstyleliteralstrong{\sphinxupquote{shared\_array}} \textendash{} 

\end{itemize}

\end{description}\end{quote}

\end{fulllineitems}

\index{save\_to\_map() (dans le module src.clusterizer.clusterizer)@\spxentry{save\_to\_map()}\spxextra{dans le module src.clusterizer.clusterizer}}

\begin{fulllineitems}
\phantomsection\label{\detokenize{index:src.clusterizer.clusterizer.save_to_map}}\pysiglinewithargsret{\sphinxcode{\sphinxupquote{src.clusterizer.clusterizer.}}\sphinxbfcode{\sphinxupquote{save\_to\_map}}}{\emph{\DUrole{n}{df\_clusters}\DUrole{p}{:} \DUrole{n}{pandas.core.frame.DataFrame}}, \emph{\DUrole{n}{map}\DUrole{p}{:} \DUrole{n}{Optional\DUrole{p}{{[}}folium.folium.Map\DUrole{p}{{]}}} \DUrole{o}{=} \DUrole{default_value}{None}}}{{ $\rightarrow$ folium.folium.Map}}
\sphinxAtStartPar
Sauvegarde les informations des clusters dans une carte Leaflet.
Retourne la carte
\begin{quote}\begin{description}
\item[{Paramètres}] \leavevmode\begin{itemize}
\item {} 
\sphinxAtStartPar
\sphinxstyleliteralstrong{\sphinxupquote{df\_clusters}} \textendash{} La DataFrame contenant les informations de chaque cluster
(cf. deuxième sortie de la fonction clusterize)

\item {} 
\sphinxAtStartPar
\sphinxstyleliteralstrong{\sphinxupquote{map}} \textendash{} la carte à utiliser
si un paramètre est spécifié : réecrit par dessus.
si rien n’est spécifié, génère une nouvelle carte

\end{itemize}

\end{description}\end{quote}

\sphinxAtStartPar
:return une carte complétée.

\end{fulllineitems}

\index{test\_geojson() (dans le module src.clusterizer.clusterizer)@\spxentry{test\_geojson()}\spxextra{dans le module src.clusterizer.clusterizer}}

\begin{fulllineitems}
\phantomsection\label{\detokenize{index:src.clusterizer.clusterizer.test_geojson}}\pysiglinewithargsret{\sphinxcode{\sphinxupquote{src.clusterizer.clusterizer.}}\sphinxbfcode{\sphinxupquote{test\_geojson}}}{}{}
\sphinxAtStartPar
Fonction interne (utilisée pour vérifier le bon fonctionnement de la clusterisation).

\end{fulllineitems}

\index{test\_naf() (dans le module src.clusterizer.clusterizer)@\spxentry{test\_naf()}\spxextra{dans le module src.clusterizer.clusterizer}}

\begin{fulllineitems}
\phantomsection\label{\detokenize{index:src.clusterizer.clusterizer.test_naf}}\pysiglinewithargsret{\sphinxcode{\sphinxupquote{src.clusterizer.clusterizer.}}\sphinxbfcode{\sphinxupquote{test\_naf}}}{}{}
\sphinxAtStartPar
Fonction interne (utilisée pour vérifier le bon fonctionnement du filtrage par NAF).

\end{fulllineitems}



\section{Fichiers utilitaires}
\label{\detokenize{index:fichiers-utilitaires}}

\subsection{Utilitaires pour la clusterisation}
\label{\detokenize{index:module-src.clusterizer.utils.clusterizer_utils}}\label{\detokenize{index:utilitaires-pour-la-clusterisation}}\index{module@\spxentry{module}!src.clusterizer.utils.clusterizer\_utils@\spxentry{src.clusterizer.utils.clusterizer\_utils}}\index{src.clusterizer.utils.clusterizer\_utils@\spxentry{src.clusterizer.utils.clusterizer\_utils}!module@\spxentry{module}}
\sphinxAtStartPar
Ce module permet d’extraire simplement nos données des GeoDataFrames, de trouver
leurs coordonnées, de restreindre le calcul aux points situés dans un certain rayon
autour de Paris ; il permet également de manipuler les clusters, de calculer leur 
poids et leur taille.
\index{calculer\_poids\_cluster() (dans le module src.clusterizer.utils.clusterizer\_utils)@\spxentry{calculer\_poids\_cluster()}\spxextra{dans le module src.clusterizer.utils.clusterizer\_utils}}

\begin{fulllineitems}
\phantomsection\label{\detokenize{index:src.clusterizer.utils.clusterizer_utils.calculer_poids_cluster}}\pysiglinewithargsret{\sphinxcode{\sphinxupquote{src.clusterizer.utils.clusterizer\_utils.}}\sphinxbfcode{\sphinxupquote{calculer\_poids\_cluster}}}{\emph{\DUrole{n}{df}\DUrole{p}{:} \DUrole{n}{pandas.core.frame.DataFrame}}, \emph{\DUrole{n}{naf\_column\_name}\DUrole{p}{:} \DUrole{n}{str}}}{{ $\rightarrow$ int}}
\sphinxAtStartPar
Calcule le poids d’un ensemble d’établissements.
\begin{quote}\begin{description}
\item[{Paramètres}] \leavevmode\begin{itemize}
\item {} 
\sphinxAtStartPar
\sphinxstyleliteralstrong{\sphinxupquote{df}} \textendash{} La DataFrame contenant tous les établissements.
Rien n’est requis, à part avoir une colonne où sont situés les codes NAF.

\item {} 
\sphinxAtStartPar
\sphinxstyleliteralstrong{\sphinxupquote{naf\_column\_name}} \textendash{} Le nom de la colonne contenant les codes NAF.

\end{itemize}

\item[{Renvoie}] \leavevmode
\sphinxAtStartPar
Le poids du cluster.

\end{description}\end{quote}

\end{fulllineitems}

\index{calculer\_poids\_cluster\_wrapper() (dans le module src.clusterizer.utils.clusterizer\_utils)@\spxentry{calculer\_poids\_cluster\_wrapper()}\spxextra{dans le module src.clusterizer.utils.clusterizer\_utils}}

\begin{fulllineitems}
\phantomsection\label{\detokenize{index:src.clusterizer.utils.clusterizer_utils.calculer_poids_cluster_wrapper}}\pysiglinewithargsret{\sphinxcode{\sphinxupquote{src.clusterizer.utils.clusterizer\_utils.}}\sphinxbfcode{\sphinxupquote{calculer\_poids\_cluster\_wrapper}}}{\emph{\DUrole{n}{naf\_column\_name}\DUrole{p}{:} \DUrole{n}{str}}}{{ $\rightarrow$ Callable\DUrole{p}{{[}}\DUrole{p}{{[}}pandas.core.frame.DataFrame\DUrole{p}{, }str\DUrole{p}{{]}}\DUrole{p}{, }int\DUrole{p}{{]}}}}
\sphinxAtStartPar
Wrappe calculer\_poids\_cluster pour pouvoir l’utiliser dans un groupby.
\begin{quote}\begin{description}
\item[{Paramètres}] \leavevmode
\sphinxAtStartPar
\sphinxstyleliteralstrong{\sphinxupquote{naf\_column\_name}} \textendash{} La colonne où se situent les codes NAF.

\item[{Renvoie}] \leavevmode
\sphinxAtStartPar
cf. la fonction calculer\_poids\_cluster.

\end{description}\end{quote}

\end{fulllineitems}

\index{calculer\_poids\_code\_NAF() (dans le module src.clusterizer.utils.clusterizer\_utils)@\spxentry{calculer\_poids\_code\_NAF()}\spxextra{dans le module src.clusterizer.utils.clusterizer\_utils}}

\begin{fulllineitems}
\phantomsection\label{\detokenize{index:src.clusterizer.utils.clusterizer_utils.calculer_poids_code_NAF}}\pysiglinewithargsret{\sphinxcode{\sphinxupquote{src.clusterizer.utils.clusterizer\_utils.}}\sphinxbfcode{\sphinxupquote{calculer\_poids\_code\_NAF}}}{\emph{\DUrole{n}{code\_naf}\DUrole{p}{:} \DUrole{n}{str}}}{{ $\rightarrow$ int}}
\sphinxAtStartPar
Calcule le poids d’un code NAF.
\begin{quote}\begin{description}
\item[{Paramètres}] \leavevmode
\sphinxAtStartPar
\sphinxstyleliteralstrong{\sphinxupquote{code\_naf}} \textendash{} Le code NAF à calculer (dans une des deux conventions : avec ou sans points).

\item[{Renvoie}] \leavevmode
\sphinxAtStartPar
Le poids du code NAF.

\end{description}\end{quote}

\end{fulllineitems}

\index{filter\_nearby\_paris() (dans le module src.clusterizer.utils.clusterizer\_utils)@\spxentry{filter\_nearby\_paris()}\spxextra{dans le module src.clusterizer.utils.clusterizer\_utils}}

\begin{fulllineitems}
\phantomsection\label{\detokenize{index:src.clusterizer.utils.clusterizer_utils.filter_nearby_paris}}\pysiglinewithargsret{\sphinxcode{\sphinxupquote{src.clusterizer.utils.clusterizer\_utils.}}\sphinxbfcode{\sphinxupquote{filter\_nearby\_paris}}}{\emph{\DUrole{n}{df}\DUrole{p}{:} \DUrole{n}{pandas.core.frame.DataFrame}}, \emph{\DUrole{n}{radius}\DUrole{p}{:} \DUrole{n}{int}}, \emph{\DUrole{n}{column\_geometry}\DUrole{p}{:} \DUrole{n}{str} \DUrole{o}{=} \DUrole{default_value}{\textquotesingle{}geometry\textquotesingle{}}}, \emph{\DUrole{n}{is\_dict}\DUrole{p}{:} \DUrole{n}{bool} \DUrole{o}{=} \DUrole{default_value}{False}}}{{ $\rightarrow$ pandas.core.frame.DataFrame}}
\sphinxAtStartPar
Filtre les données proches du centre de Paris.
\begin{quote}\begin{description}
\item[{Paramètres}] \leavevmode\begin{itemize}
\item {} 
\sphinxAtStartPar
\sphinxstyleliteralstrong{\sphinxupquote{df}} \textendash{} la DataFrame à filtrer

\item {} 
\sphinxAtStartPar
\sphinxstyleliteralstrong{\sphinxupquote{radius}} \textendash{} le rayon (en kilomètres)

\item {} 
\sphinxAtStartPar
\sphinxstyleliteralstrong{\sphinxupquote{column\_geometry}} \textendash{} la colonne où se trouvent les données géométriques (par défaut : “geometry”)

\end{itemize}

\item[{Renvoie}] \leavevmode
\sphinxAtStartPar
la DataFrame filtrée

\end{description}\end{quote}

\end{fulllineitems}

\index{get\_coords\_from\_object() (dans le module src.clusterizer.utils.clusterizer\_utils)@\spxentry{get\_coords\_from\_object()}\spxextra{dans le module src.clusterizer.utils.clusterizer\_utils}}

\begin{fulllineitems}
\phantomsection\label{\detokenize{index:src.clusterizer.utils.clusterizer_utils.get_coords_from_object}}\pysiglinewithargsret{\sphinxcode{\sphinxupquote{src.clusterizer.utils.clusterizer\_utils.}}\sphinxbfcode{\sphinxupquote{get\_coords\_from\_object}}}{\emph{\DUrole{n}{df}\DUrole{p}{:} \DUrole{n}{pandas.core.frame.DataFrame}}, \emph{\DUrole{n}{column\_geometry}\DUrole{p}{:} \DUrole{n}{str} \DUrole{o}{=} \DUrole{default_value}{\textquotesingle{}geometry\textquotesingle{}}}, \emph{\DUrole{n}{is\_dict}\DUrole{p}{:} \DUrole{n}{bool} \DUrole{o}{=} \DUrole{default_value}{False}}}{{ $\rightarrow$ numpy.ndarray}}
\sphinxAtStartPar
Récupère les coordonnées des points de la DataFrame.
\begin{quote}\begin{description}
\item[{Paramètres}] \leavevmode\begin{itemize}
\item {} 
\sphinxAtStartPar
\sphinxstyleliteralstrong{\sphinxupquote{df}} \textendash{} la DataFrame.

\item {} 
\sphinxAtStartPar
\sphinxstyleliteralstrong{\sphinxupquote{column\_geometry}} \textendash{} la colonne contenant les données géométriques.

\item {} 
\sphinxAtStartPar
\sphinxstyleliteralstrong{\sphinxupquote{is\_dict}} \textendash{} les données sont\sphinxhyphen{}elles en dictionnaire ?

\end{itemize}

\item[{Renvoie}] \leavevmode
\sphinxAtStartPar
les coordonnées sous la forme d’une matrice de deux colonnes (et d’autant de lignes qu’il y a de points)

\end{description}\end{quote}

\end{fulllineitems}

\index{get\_infos\_clusters\_enveloppes\_convexes() (dans le module src.clusterizer.utils.clusterizer\_utils)@\spxentry{get\_infos\_clusters\_enveloppes\_convexes()}\spxextra{dans le module src.clusterizer.utils.clusterizer\_utils}}

\begin{fulllineitems}
\phantomsection\label{\detokenize{index:src.clusterizer.utils.clusterizer_utils.get_infos_clusters_enveloppes_convexes}}\pysiglinewithargsret{\sphinxcode{\sphinxupquote{src.clusterizer.utils.clusterizer\_utils.}}\sphinxbfcode{\sphinxupquote{get\_infos\_clusters\_enveloppes\_convexes}}}{\emph{\DUrole{n}{k}\DUrole{p}{:} \DUrole{n}{int}}, \emph{\DUrole{n}{df}\DUrole{p}{:} \DUrole{n}{pandas.core.frame.DataFrame}}, \emph{\DUrole{n}{column\_geometry}\DUrole{p}{:} \DUrole{n}{str} \DUrole{o}{=} \DUrole{default_value}{\textquotesingle{}geometry\textquotesingle{}}}, \emph{\DUrole{n}{is\_dict}\DUrole{p}{:} \DUrole{n}{bool} \DUrole{o}{=} \DUrole{default_value}{False}}}{{ $\rightarrow$ pandas.core.frame.DataFrame}}
\sphinxAtStartPar
Fonction permettant de récupérer des infos sur les clusters (enveloppes convexes).
\begin{quote}\begin{description}
\item[{Paramètres}] \leavevmode\begin{itemize}
\item {} 
\sphinxAtStartPar
\sphinxstyleliteralstrong{\sphinxupquote{k}} \textendash{} Nombre de clusters

\item {} 
\sphinxAtStartPar
\sphinxstyleliteralstrong{\sphinxupquote{df}} \textendash{} La DataFrame où l’on a déjà ajouté le numéro des clusters (laissée intacte).

\item {} 
\sphinxAtStartPar
\sphinxstyleliteralstrong{\sphinxupquote{column\_geometry}} \textendash{} Le nom de la colonne où se situent les données géometriques (par défaut, « geometry »).

\item {} 
\sphinxAtStartPar
\sphinxstyleliteralstrong{\sphinxupquote{is\_dict}} \textendash{} True si les paramètres sont sous forme de dictionnaire

\end{itemize}

\item[{Renvoie}] \leavevmode
\sphinxAtStartPar
Une GeoDataFrame associant à chaque numéro de cluster son enveloppe convexe.

\end{description}\end{quote}

\end{fulllineitems}

\index{get\_infos\_clusters\_poids() (dans le module src.clusterizer.utils.clusterizer\_utils)@\spxentry{get\_infos\_clusters\_poids()}\spxextra{dans le module src.clusterizer.utils.clusterizer\_utils}}

\begin{fulllineitems}
\phantomsection\label{\detokenize{index:src.clusterizer.utils.clusterizer_utils.get_infos_clusters_poids}}\pysiglinewithargsret{\sphinxcode{\sphinxupquote{src.clusterizer.utils.clusterizer\_utils.}}\sphinxbfcode{\sphinxupquote{get\_infos\_clusters\_poids}}}{\emph{\DUrole{n}{df}\DUrole{p}{:} \DUrole{n}{pandas.core.frame.DataFrame}}, \emph{\DUrole{n}{column\_naf\_code}\DUrole{p}{:} \DUrole{n}{str}}}{{ $\rightarrow$ pandas.core.frame.DataFrame}}
\sphinxAtStartPar
Fonction permettant de récupérer des infos sur les clusters (poids).
\begin{quote}\begin{description}
\item[{Paramètres}] \leavevmode\begin{itemize}
\item {} 
\sphinxAtStartPar
\sphinxstyleliteralstrong{\sphinxupquote{df}} \textendash{} La DataFrame où l’on a déjà ajouté le numéro des clusters (laissée intacte).

\item {} 
\sphinxAtStartPar
\sphinxstyleliteralstrong{\sphinxupquote{column\_naf\_code}} \textendash{} Le nom de la colonne où se situent les codes NAF.

\end{itemize}

\item[{Renvoie}] \leavevmode
\sphinxAtStartPar
Une nouvelle GeoDataFrame associant à chaque numéro de cluster le poids de celui\sphinxhyphen{}ci

\end{description}\end{quote}

\end{fulllineitems}

\index{get\_infos\_clusters\_taille() (dans le module src.clusterizer.utils.clusterizer\_utils)@\spxentry{get\_infos\_clusters\_taille()}\spxextra{dans le module src.clusterizer.utils.clusterizer\_utils}}

\begin{fulllineitems}
\phantomsection\label{\detokenize{index:src.clusterizer.utils.clusterizer_utils.get_infos_clusters_taille}}\pysiglinewithargsret{\sphinxcode{\sphinxupquote{src.clusterizer.utils.clusterizer\_utils.}}\sphinxbfcode{\sphinxupquote{get\_infos\_clusters\_taille}}}{\emph{\DUrole{n}{df}\DUrole{p}{:} \DUrole{n}{pandas.core.frame.DataFrame}}}{{ $\rightarrow$ pandas.core.frame.DataFrame}}
\sphinxAtStartPar
Fonction permettant de récupérer des infos sur les clusters (tailles).
\begin{quote}\begin{description}
\item[{Paramètres}] \leavevmode
\sphinxAtStartPar
\sphinxstyleliteralstrong{\sphinxupquote{df}} \textendash{} La DataFrame où l’on a déjà ajouté le numéro des clusters (laissée intacte).

\item[{Renvoie}] \leavevmode
\sphinxAtStartPar
Une nouvelle GeoDataFrame associant à chaque numéro de cluster la taille de celui\sphinxhyphen{}ci
(nombre d’établissements)

\end{description}\end{quote}

\end{fulllineitems}

\index{swap\_xy() (dans le module src.clusterizer.utils.clusterizer\_utils)@\spxentry{swap\_xy()}\spxextra{dans le module src.clusterizer.utils.clusterizer\_utils}}

\begin{fulllineitems}
\phantomsection\label{\detokenize{index:src.clusterizer.utils.clusterizer_utils.swap_xy}}\pysiglinewithargsret{\sphinxcode{\sphinxupquote{src.clusterizer.utils.clusterizer\_utils.}}\sphinxbfcode{\sphinxupquote{swap\_xy}}}{\emph{\DUrole{n}{geom}}}{}
\sphinxAtStartPar
Inverse les coordonnées de l’objet shapely.geometry.
Utile pour passer objets shapely dans folium (la convention est inversée).
Auteur : \sphinxurl{https://gis.stackexchange.com/a/291293}
\begin{quote}\begin{description}
\item[{Paramètres}] \leavevmode
\sphinxAtStartPar
\sphinxstyleliteralstrong{\sphinxupquote{geom}} \textendash{} L’objet dont on veut inverser les coordonnées (Point, Polygon, MultiPolygon, etc.)

\item[{Renvoie}] \leavevmode
\sphinxAtStartPar
l’objet inversé

\end{description}\end{quote}

\end{fulllineitems}



\subsection{Utilitaires pour la gestion des codes NAF}
\label{\detokenize{index:module-src.clusterizer.utils.NAF_utils}}\label{\detokenize{index:utilitaires-pour-la-gestion-des-codes-naf}}\index{module@\spxentry{module}!src.clusterizer.utils.NAF\_utils@\spxentry{src.clusterizer.utils.NAF\_utils}}\index{src.clusterizer.utils.NAF\_utils@\spxentry{src.clusterizer.utils.NAF\_utils}!module@\spxentry{module}}
\sphinxAtStartPar
Fonctions pour switcher les conventions de NAF (avec ou sans point intermédiaire)
\index{ajouter\_point() (dans le module src.clusterizer.utils.NAF\_utils)@\spxentry{ajouter\_point()}\spxextra{dans le module src.clusterizer.utils.NAF\_utils}}

\begin{fulllineitems}
\phantomsection\label{\detokenize{index:src.clusterizer.utils.NAF_utils.ajouter_point}}\pysiglinewithargsret{\sphinxcode{\sphinxupquote{src.clusterizer.utils.NAF\_utils.}}\sphinxbfcode{\sphinxupquote{ajouter\_point}}}{\emph{\DUrole{n}{code\_naf}\DUrole{p}{:} \DUrole{n}{str}}}{{ $\rightarrow$ Optional\DUrole{p}{{[}}str\DUrole{p}{{]}}}}
\sphinxAtStartPar
Fait passer le code NAF à la convention avec point (s’il n’y est pas)
\begin{quote}\begin{description}
\item[{Paramètres}] \leavevmode
\sphinxAtStartPar
\sphinxstyleliteralstrong{\sphinxupquote{code\_naf}} \textendash{} Le code à changer

\item[{Renvoie}] \leavevmode
\sphinxAtStartPar
Le code avec un point.

\end{description}\end{quote}

\end{fulllineitems}

\index{filter\_by\_naf() (dans le module src.clusterizer.utils.NAF\_utils)@\spxentry{filter\_by\_naf()}\spxextra{dans le module src.clusterizer.utils.NAF\_utils}}

\begin{fulllineitems}
\phantomsection\label{\detokenize{index:src.clusterizer.utils.NAF_utils.filter_by_naf}}\pysiglinewithargsret{\sphinxcode{\sphinxupquote{src.clusterizer.utils.NAF\_utils.}}\sphinxbfcode{\sphinxupquote{filter\_by\_naf}}}{\emph{\DUrole{n}{df}\DUrole{p}{:} \DUrole{n}{pandas.core.frame.DataFrame}}, \emph{\DUrole{n}{codes\_naf}\DUrole{p}{:} \DUrole{n}{List\DUrole{p}{{[}}str\DUrole{p}{{]}}}}, \emph{\DUrole{n}{column\_codes}\DUrole{p}{:} \DUrole{n}{str}}}{{ $\rightarrow$ pandas.core.frame.DataFrame}}
\sphinxAtStartPar
Retourne les établissements dont le code NAF est contenu dans la liste.
\begin{quote}\begin{description}
\item[{Paramètres}] \leavevmode\begin{itemize}
\item {} 
\sphinxAtStartPar
\sphinxstyleliteralstrong{\sphinxupquote{df}} \textendash{} La liste des établissements (convention NAF : sans le point)

\item {} 
\sphinxAtStartPar
\sphinxstyleliteralstrong{\sphinxupquote{codes\_naf}} \textendash{} Les codes NAF (avec ou sans le point) (sous forme de liste)

\item {} 
\sphinxAtStartPar
\sphinxstyleliteralstrong{\sphinxupquote{column\_codes}} \textendash{} La colonne où est située le code NAF dans la DataFrame des établissements

\end{itemize}

\item[{Renvoie}] \leavevmode
\sphinxAtStartPar
La DataFrame filtrée.

\end{description}\end{quote}

\end{fulllineitems}

\index{get\_NAFs\_by\_section() (dans le module src.clusterizer.utils.NAF\_utils)@\spxentry{get\_NAFs\_by\_section()}\spxextra{dans le module src.clusterizer.utils.NAF\_utils}}

\begin{fulllineitems}
\phantomsection\label{\detokenize{index:src.clusterizer.utils.NAF_utils.get_NAFs_by_section}}\pysiglinewithargsret{\sphinxcode{\sphinxupquote{src.clusterizer.utils.NAF\_utils.}}\sphinxbfcode{\sphinxupquote{get\_NAFs\_by\_section}}}{\emph{\DUrole{n}{section}\DUrole{p}{:} \DUrole{n}{str}}}{{ $\rightarrow$ pandas.core.series.Series}}
\sphinxAtStartPar
Fournit la liste des codes NAF de la section correspondante.
\begin{quote}\begin{description}
\item[{Paramètres}] \leavevmode
\sphinxAtStartPar
\sphinxstyleliteralstrong{\sphinxupquote{section}} \textendash{} La lettre de la section

\item[{Renvoie}] \leavevmode
\sphinxAtStartPar
La liste des codes NAF contenus dans la section (convention : avec points)

\end{description}\end{quote}

\end{fulllineitems}

\index{get\_description() (dans le module src.clusterizer.utils.NAF\_utils)@\spxentry{get\_description()}\spxextra{dans le module src.clusterizer.utils.NAF\_utils}}

\begin{fulllineitems}
\phantomsection\label{\detokenize{index:src.clusterizer.utils.NAF_utils.get_description}}\pysiglinewithargsret{\sphinxcode{\sphinxupquote{src.clusterizer.utils.NAF\_utils.}}\sphinxbfcode{\sphinxupquote{get\_description}}}{\emph{\DUrole{n}{code\_naf}\DUrole{p}{:} \DUrole{n}{str}}}{{ $\rightarrow$ str}}
\sphinxAtStartPar
Fournit la description correspondant au code NAF.
\begin{quote}\begin{description}
\item[{Paramètres}] \leavevmode
\sphinxAtStartPar
\sphinxstyleliteralstrong{\sphinxupquote{code\_naf}} \textendash{} le code, avec ou sans point.

\item[{Renvoie}] \leavevmode
\sphinxAtStartPar
la description complète.

\end{description}\end{quote}

\end{fulllineitems}

\index{retirer\_point() (dans le module src.clusterizer.utils.NAF\_utils)@\spxentry{retirer\_point()}\spxextra{dans le module src.clusterizer.utils.NAF\_utils}}

\begin{fulllineitems}
\phantomsection\label{\detokenize{index:src.clusterizer.utils.NAF_utils.retirer_point}}\pysiglinewithargsret{\sphinxcode{\sphinxupquote{src.clusterizer.utils.NAF\_utils.}}\sphinxbfcode{\sphinxupquote{retirer\_point}}}{\emph{\DUrole{n}{code\_naf}\DUrole{p}{:} \DUrole{n}{str}}}{{ $\rightarrow$ Optional\DUrole{p}{{[}}str\DUrole{p}{{]}}}}
\sphinxAtStartPar
Fait passer le code NAF à la convention sans point (s’il y est)
\begin{quote}\begin{description}
\item[{Paramètres}] \leavevmode
\sphinxAtStartPar
\sphinxstyleliteralstrong{\sphinxupquote{code\_naf}} \textendash{} Le code à changer

\item[{Renvoie}] \leavevmode
\sphinxAtStartPar
Le code sans point.

\end{description}\end{quote}

\end{fulllineitems}



\chapter{Traitement de la base SIRENE}
\label{\detokenize{index:traitement-de-la-base-sirene}}

\chapter{Indices and tables}
\label{\detokenize{index:indices-and-tables}}\begin{itemize}
\item {} 
\sphinxAtStartPar
\DUrole{xref,std,std-ref}{genindex}

\item {} 
\sphinxAtStartPar
\DUrole{xref,std,std-ref}{modindex}

\item {} 
\sphinxAtStartPar
\DUrole{xref,std,std-ref}{search}

\end{itemize}


\renewcommand{\indexname}{Index des modules Python}
\begin{sphinxtheindex}
\let\bigletter\sphinxstyleindexlettergroup
\bigletter{s}
\item\relax\sphinxstyleindexentry{src.clusterizer.clusterizer}\sphinxstyleindexpageref{index:\detokenize{module-src.clusterizer.clusterizer}}
\item\relax\sphinxstyleindexentry{src.clusterizer.utils.clusterizer\_utils}\sphinxstyleindexpageref{index:\detokenize{module-src.clusterizer.utils.clusterizer_utils}}
\item\relax\sphinxstyleindexentry{src.clusterizer.utils.NAF\_utils}\sphinxstyleindexpageref{index:\detokenize{module-src.clusterizer.utils.NAF_utils}}
\item\relax\sphinxstyleindexentry{src.ihm.ihm\_complet}\sphinxstyleindexpageref{index:\detokenize{module-src.ihm.ihm_complet}}
\item\relax\sphinxstyleindexentry{src.ihm.ihm\_csv}\sphinxstyleindexpageref{index:\detokenize{module-src.ihm.ihm_csv}}
\item\relax\sphinxstyleindexentry{src.ihm.ihm\_pyqt}\sphinxstyleindexpageref{index:\detokenize{module-src.ihm.ihm_pyqt}}
\item\relax\sphinxstyleindexentry{src.ihm.web}\sphinxstyleindexpageref{index:\detokenize{module-src.ihm.web}}
\end{sphinxtheindex}

\renewcommand{\indexname}{Index}
\printindex
\end{document}